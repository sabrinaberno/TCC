\chapter[Introdução]{Introdução}

\section{Contexto}

A fertilização in vitro (FIV) é uma das técnicas mais importantes de reprodução assistida, e tem ganhado crescente relevância no campo da medicina reprodutiva, oferecendo novas possibilidades para pessoas com dificuldades para engravidar. Maria Chaves Jardim destaca que o Brasil lidera a América Latina em número de procedimentos de fertilização in vitro, evidenciando a relevância e o avanço dessa tecnologia no país.\cite{jardim2023}. 

O sucesso da FIV está diretamente relacionado à saúde genética dos embriões utilizados. De acordo com \citeonline{ping2023}, a qualidade genética dos embriões é um fator crucial para o êxito da fertilização, uma vez que aneuploidias estão frequentemente associadas a desfechos desfavoráveis, como abortos espontâneos ou doenças genéticas nos fetos. Cerca de 50\% dos casos de abortos espontâneos no primeiro trimestre estão relacionados a alterações cromossômicas \cite{silva2023}. Este cenário destaca a importância de métodos de avaliação genética, que se tornam ferramentas essenciais para aumentar as taxas de sucesso da FIV e melhorar a segurança dos tratamentos.

Para avaliar a qualidade dos embriões, são comumente utilizados testes genéticos de triagem embrionária, sendo o Teste Genético Pré-Implantação (PGT) um dos mais conhecidos e amplamente aplicados, especialmente o Teste Genético Pré-Implantação de Aneuploidia (PGT-A) \cite{yang2024}. No entanto, a realização do PGT-A apresenta alguns desafios e limitações, como o tempo necessário para obtenção dos resultados, o custo elevado e a complexidade e o risco do procedimento, que envolve a biópsia embrionária (extração de células do trofectoderma) e, por isso, pode causar daoo ao embrião \cite{yang2024}. Assim, surge a oportunidade de se desenvolverum método de avaliação embrionária mais rápido, menos custoso e menos invasivo.

Um exemplo de tecnologia que pode unir Tecnologia de Reprodução Assistida (TRA) e Inteligência Artificial (IA) é a fotografia time-lapse incorporada às incubadoras de última geração, que geram informações sobre a morfologia e a cinética do desenvolvimento embrionário, e facilita a observação de eventos dinâmicos, seus tempos e padrões, definindo-os, em conjunto, como variáveis morfocinéticas \cite{meseguer2011}. Postula-se que, com o maior conhecimento da dinâmica embrionária em cultivo, a identificação de marcadores do potencial de implantação poderá fornecer, em futuro próximo, informações cruciais para o processo de escolha do embrião a ser transferido para o útero materno \cite{luong2023}. Entretanto, até aqui, pouco se evoluiu no que diz respeito à aplicação dos dados ou à análise de variáveis múltiplas sobre resultados clínicos objetivos.

A proposta deste projeto é desenvolver uma abordagem de baixo custo, utilizando IA para a detecção de padrões genéticos em embriões. Mais especificamente, o objetivo é identificar a probabilidade de um embrião ser euploide, isto é, aquele que possui a quantidade correta de cromossomos (23 pares), ou seja, os cromossomos estão organizados em pares completos \cite{zegers2017}. Ao realizar a análise dos dados gerados pelo TLS, a IA poderá oferecer uma abordagem menos invasiva, eliminando a necessidade de testes genéticos caros e invasivos, como o PGT-A. Assim, ao melhorar a seleção dos embriões com maior potencial de euploidia, a IA poderá tornar o tratamento mais acessível para um maior número de pacientes e, no futuro, aumentar as taxas de sucesso da FIV.

Para as clínicas de fertilização, a implementação dessa tecnologia pode resultar em um fluxo de trabalho mais ágil e eficiente, otimizando recursos e melhorando a satisfação dos pacientes. A integração da IA na avaliação embrionária tem o potencial de transformar os tratamentos de fertilidade, proporcionando resultados mais positivos e uma experiência aprimorada para todos os envolvidos. Assim, este estudo busca ser uma solução menos invasiva dentre os métodos de seleção embrionária em tratamentos de reprodução assistida, integrando as inovações da IA à medicina reprodutiva.

\section{Motivação}

A motivação para a elaboração deste Trabalho de Conclusão de Curso surge da necessidade de contruibir para a criação de métodos mais acessíveis na medicina reprodutiva, especialmente no que diz respeito à avaliação da qualidade genética dos embriões em tratamentos de fertilidade. Essa área, que lida com questões pessoais e delicadas, fez avanços notáveis na última década \cite{pandit2022}, mas ainda enfrenta desafios significativos que exigem soluções eficazes. O avanço das tecnologias de reprodução assistida possibilitou que muitas pessoas, antes incapazes de conceber, realizassem o sonho da gravidez. No entanto, essa evolução trouxe consigo um novo desafio: a falha na implantação e a gravidez não viável. A repetição de ciclos sem sucesso pode gerar profunda frustração e desespero, fazendo com que os casais busquem incessantemente respostas e soluções \cite{montagnini2010}.

A jornada para a maternidade através da FIV é marcada por uma intensa carga emocional, com expectativas e incertezas. A perda gestacional, além do sofrimento emocional, pode gerar um impacto significativo na saúde física e mental da mulher  \cite{montagnini2010}. De acordo com o estudo de \citeonline{montagnini2010}, as mulheres apresentam, em comparação aos homens, níveis mais elevados de ansiedade e depressão, além de uma autoestima mais baixa, com sentimentos de culpa e vergonha relacionados à infertilidade. A ansiedade, em particular, é um dos principais desafios enfrentados pelos casais, frequentemente ultrapassando níveis considerados normais \cite{montagnini2010}.

A escolha cuidadosa dos embriões para a transferência é um passo crucial nesse processo de tentativa de adicionar um membro à família, pois a qualidade e a saúde dos embriões influenciam diretamente a taxa de sucesso da gravidez \cite{yang2024}. O impacto da seleção de embriões com defeitos genéticos pode ser devastador para o emocional dos casais, agravando o sofrimento causado por abortos espontâneos e falhas de implantação. Ao eliminar a necessidade de intervenções, como o exame PGT-A, que podem gerar ansiedade e desconforto, podemos proporcionar um ambiente mais acolhedor e propício à realização do sonho de ter filhos.

Portanto, a relevância deste projeto no contexto atual da medicina reprodutiva é indiscutível. Ao abordar as limitações dos métodos tradicionais e explorar as potencialidades da IA, este estudo permite que os médicos tomem decisões mais informadas, com base em análises detalhadas, resultando em melhores resultados clínicos. Além disso, este trabalho destaca a importância da Engenharia de Software como uma ferramenta essencial para impulsionar o desenvolvimento de outras áreas do conhecimento. Ao aplicar técnicas avançadas de análise de dados e algoritmos de Machine Learning, ele contribui diretamente para o crescimento da medicina reprodutiva, promovendo inovações que tornam os tratamentos mais precisos, acessíveis e eficientes. A integração de abordagens interdisciplinares e a personalização do tratamento têm o potencial de gerar avanços significativos em médio e longo prazo, além de reduzir o tempo necessário para se obter um nascimento saudável. Assim, este projeto não apenas beneficia a área médica, mas também reforça o papel da Engenharia de Software como transformadora em diferentes campos científicos.

\section{Problema}

A seleção de embriões euploides em procedimentos de fertilização in vitro (FIV) é determinante para o sucesso do tratamento. No entanto, esse processo ainda depende de técnicas invasivas, como o Teste Genético Pré-Implantacional de Aneuploidia, um procedimento relativamente complexo, pois requer uma biópsia do embrião, durante a qual deve ser garantido o mínimo de danos ao embrião \cite{yang2024}.

Dessa forma, o objetivo deste trabalho é responder à seguinte pergunta: "Como utilizar a inteligência artificial para identificar padrões em dados morfocinéticos de embriões, obtidos por meio do Time-Lapse System, para prever a porcentagem de euploidia, oferecendo uma solução mais eficaz e menos invasiva do que o PGT-A?"

\section{Objetivos}

\subsection{Objetivos Gerais}
Desenvolver uma abordagem baseada em inteligência artificial para identificar padrões em dados morfocinéticos de embriões, obtidos por meio do Time-Lapse System, capaz de predizer a porcentagem de euploidia, proporcionando uma solução mais eficaz e menos invasiva em comparação ao PGT-A.

\subsection{Objetivos Específicos}
\begin{itemize}
    \item \textbf{OE1}: Expansão, Processamento e Análise de Dados para Predição de Ploidia.
    \item \textbf{OE2}: Treinamento e Ajuste de Modelo de Machine Learning para Predição de Euploidia.
    \item \textbf{OE3}: Avaliação do Modelo
    \item \textbf{OE4}: Protótipo de Interface
\end{itemize}

\section{Metodologia}
A metodologia do projeto está dividida em 3 fases, onde cada uma visa resolver um Objetivo Específico (OE) do projeto, que contém suas respectivas atividades para serem alcançados:

\begin{itemize}
    \item \textbf{Fase 1: Análise e Preparação de Dados}
    \begin{itemize}
        \item \textbf{OE1 - Expansão, Processamento e Análise de Dados para Predição de Ploidia}: 
        \begin{itemize}
            \item \textbf{Atividade 1 (A1):} Análise, Revisão, Seleção e Limpeza de Variáveis para Predição de Euploidia 
            \item \textbf{Atividade 2 (A2):} Normalização dos Dados para Otimização.
            \item \textbf{Atividade 3 (A3):} Identificação da Correlação e Atribuição de Pesos aos Parâmetros na Previsão da Ploidia do Embrião 
            \item \textbf{Atividade 4 (A4):} Divisão dos Dados e aplicação de Data Augmentation
        \end{itemize}
    \end{itemize}

    \item \textbf{Fase 2: Desenvolvimento e Avaliação do Modelo}
    \begin{itemize}
        \item \textbf{OE2 - Treinamento e Ajuste de Modelo de Machine Learning para Predição de Euploidia}: 
        \begin{itemize}
            \item \textbf{Atividade 5 (A5):} Desenvolvimento e Treinamento do Modelo de Machine Learning para Otimização da Predição de Euploidia, Incluindo Treinamento, Validação e Teste
        \end{itemize}

        \item \textbf{OE3 - Avaliação do Modelo}: 
        \begin{itemize}
            \item \textbf{Atividade 6 (A6):} Utilizar métricas adequadas para medir o desempenho do modelo
            \item \textbf{Atividade 7 (A7):} Avaliação do Desempenho do Modelo na Predição por meio da Matriz de Confusão e Curva ROC
        \end{itemize}

        \item \textbf{OE4 - Protótipo de Interface}: 
        \begin{itemize}
            \item \textbf{Atividade 8 (A8):} Prototipar uma interface
        \end{itemize}
    \end{itemize}
\end{itemize}

\section{Composição e estrutura do trabalho}
Este trabalho foi organizado da seguinte maneira:

\textbf{Capítulo 2}, intitulado "Referencial Teórico", apresenta os principais conceitos que fundamentam a contextualização deste estudo.

\textbf{Capítulo 3}, intitulado "Metodologia", descreve os procedimentos e métodos utilizados na pesquisa, incluindo o planejamento de trabalho, as atividades realizadas e os resultados esperados. A seção 1.5 será detalhada ao longo deste capítulo.

\textbf{Capítulo 4}, intitulado "Execução da Pesquisa e Análise dos Resultados", apresenta a previsão de término de cada atividade e fase propostas ao longo do trabalho.

\textbf{Capítulo 5}, intitulado "Considerações e Trabalhos Futuros", expõe os avanços e resultados obtidos durante o período de desenvolvimento deste TCC.

\textbf{Capítulo 6}, intitulado "Planejamento", apresentas todas as atividades realizadas e as atividades futuras a serem realizadas.