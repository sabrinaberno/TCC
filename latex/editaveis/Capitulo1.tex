\chapter[Introdução]{Introdução}
\addcontentsline{toc}{chapter}{Introdução}

\section{Contexto}

Uma das técnicas mais importantes de reprodução assistida que adquiriu crescente importância no campo da medicina reprodutiva é a fertilização in vitro (FIV). Ela fornece uma nova perspectiva para pessoas que têm problemas para engravidar. De acordo com a Associação Brasileira de Reprodução Assistida (SBRA), o mercado de reprodução assistida deve se expandir a um ritmo próximo de 23\% ao ano até 2026, refletindo a crescente demanda por essa tecnologia.

O sucesso da fertilização in vitro (FIV) está ligado à saúde genética dos óvulos utilizados. De acordo com , a qualidade genética dos óvulos é um fator crucial para o resultado positivo da FIV, visto que embriões com aneuploidias estão frequentemente associados a complicações nos resultados, uma vez que podem levar a abortos espontâneos. Pesquisas recentes sugerem que um número significativo das perdas gestacionais, estima-se em até 50\%, está relacionado a tais anomalias cromossômicas . Esse contexto reforça a importância de métodos de avaliação genética como ferramenta essencial para aprimorar os índices de sucesso da FIV e aumentar a segurança dos tratamentos.

Para avaliar a qualidade dos embriões, são utilizados testes genéticos clássicos, sendo o mais conhecido e amplamente aplicado o Teste Genético Pré-Implantação (PGT), especificamente o Teste Genético Pré-Implantação para Aneuploidia (PGT-A). No entanto, esse teste levanta alguns desafios e limitações, como o tempo para ter resultado, seu custo elevado e a complexidade em termos de procedimento, que utiliza uma biópsia embrionária, onde a minimização de danos ao embrião deve ser assegurada. Portanto, surgem questões em pensar em um método de avaliação embrionária que seja mais rápido, menos custoso e menos invasivo.

Com essas lacunas de ideias para um novo método, nos dias atuais temos a possibilidade de combinar o que é desejável de mudança na FIV com a IA (Inteligência Artificial), para explorar uma técnica que pode melhorar significativamente os resultados em medicina reprodutiva. A inteligência artificial (IA) está surgindo como uma ferramenta poderosa que pode transformar as técnicas em que os diagnósticos genéticos são realizados e devemos utilizá-la ao nosso favor para chegarmos nos resultados que esperamos. A maioria dos algoritmos atualmente utilizados no mercado foca na predição da capacidade de implantação do embrião, ou seja, na probabilidade de o embrião resultar em uma gravidez. Porém, nem toda gravidez decorrente desse processo é geneticamente saudável. O que propomos neste projeto é o desenvolvimento de uma abordagem de menor custo, menos invasiva e mais rápida para a detecção de padrões genéticos em óvulos, visando identificar a probabilidade de um embrião ser euploide, desenvolvendo um algoritmo capaz de fornecer informações sobre a saúde genética do embrião, ou seja, indicar as chances de euploidia com base na combinação dos fatores analisados, com o auxílio da inteligência artificial.

Utilizaremos dados de embriões analisados por um profissional embriologista, dados esses tirados de um sistema que monitora o embrião por meio de fotos e vídeos, e assim, com os dados obtidos e transcritos pelo profissional, fornecem um suporte valioso para a identificação de diferentes características dos embriões, como o timing de divisões celulares, padrões de crescimento, entre outras, entregando dados importantes que contribuem para o processo de seleção de embriões. Com a inteligência artificial efetuando uma análise dos dados obtidos, essa técnica junto com a tecnologia, poderá oferecer uma abordagem menos invasiva, eliminando a necessidade de testes genéticos invasivos, como a PGT-A. Melhorando a seleção de embriões com maior potencial de euploidia, a IA poderá tornar o tratamento mais acessível a um maior número de pacientes e, futuramente, aumentar as taxas de sucesso em fertilização in vitro (FIV).

Para as clínicas, a implementação dessa tecnologia pode resultar em um fluxo de trabalho mais ágil e eficiente, permitindo a otimização dos recursos e a melhoria na satisfação dos pacientes. A integração da IA na avaliação de embriões tem o potencial de transformar a forma como os tratamentos de fertilidade são conduzidos, proporcionando resultados mais positivos e uma experiência melhor para todos os envolvidos. Assim, este estudo visa aprimorar os métodos de seleção de embriões em tratamentos de reprodução assistida, integrando a medicina reprodutiva com as inovações da inteligência artificial.


\section{Motivação}

A motivação para a escrita desse Trabalho de Conclusão de Curso surge da necessidade de criação de métodos mais acessíveis na medicina reprodutiva, em especial na avaliação da qualidade genética dos óvulos em tratamentos de fertilidade. Essa área, que lida com questões pessoais e delicadas, fez avanços notáveis na última década (Pandit, 2020), mas ainda necessita de soluções significativas para os desafios enfrentados. O avanço das tecnologias de reprodução assistida possibilitou que muitas pessoas, antes incapazes de conceber, realizassem o sonho de uma gravidez. No entanto, isso trouxe um novo desafio: a falha de implantação e gravidez não viável. A repetição de ciclos sem sucesso pode gerar profunda frustração e desespero, levando os casais a buscar incansavelmente por respostas e soluções.

A jornada para a maternidade através da FIV é marcada por uma intensa carga emocional, com expectativas e incertezas. A perda gestacional, além do sofrimento emocional, pode gerar um impacto significativo na saúde física e mental da mulher. De acordo com o estudo de Montagnini et al. (2020), as mulheres apresentam, em comparação aos homens, níveis mais altos de ansiedade e depressão, além de autoestima mais baixa e sentimentos de culpa e vergonha relacionados à infertilidade. A ansiedade, em particular, é um dos principais desafios enfrentados pelos casais, frequentemente excedendo níveis considerados normais. Por isso, 

A escolha cuidadosa dos embriões para a transferência é um passo crucial nesse processo de tentativa de adicionar um membro na família, pois a qualidade dos óvulos e a saúde dos embriões podem influenciar diretamente a taxa de sucesso da gravidez. O impacto da seleção de embriões com defeitos genéticos pode ser devastador para o emocional dos casais, agravando o sofrimento causado por abortos espontâneos e falhas de implantação. Ao eliminarmos a necessidade de intervenções, como o exame PGT-A, que podem causar ansiedade e desconforto, proporcionamos um ambiente mais acolhedor e propício à realização do sonho de ter filhos.

Portanto, a relevância deste projeto no contexto atual da medicina reprodutiva é indiscutível. Ao abordar as limitações dos métodos tradicionais e explorar as potencialidades da IA, este estudo permite que médicos tomem decisões mais informadas com base em análises detalhadas, resultando em melhores resultados clínicos. A integração de abordagens interdisciplinares e a personalização do tratamento podem revolucionar a prática da reprodução assistida, proporcionando benefícios significativos tanto para as clínicas quanto para os pacientes.


\section{Problema}

A gravidez é caracterizada pela presença de beta gonadotrofina coriônica humana (hCG) no sangue materno (Early pregnancy loss in IVF: a literature review). Na maioria dos ciclos de FIV humana, múltiplos embriões são criados após a hiperestimulação ovariana, mesmo assim a taxa de gravidez clínica do ciclo de fertilização in vitro é estimada acima de 60\%, enquanto a taxa real de nascidos vivos ainda está bem abaixo de 50% (Early pregnancy loss in IVF: a literature review). 

Atualmente, os casais têm a opção de testar geneticamente seus embriões para verificar se possuem determinadas doenças que podem comprometer o desenvolvimento adequado dentro do útero ou após o nascimento. A viabilidade dos embriões e, consequentemente, a probabilidade de um embrião se implantar com sucesso, são influenciadas por diversos fatores, como fatores genéticos, que desempenham um papel significativo nos abortos espontâneos, com estudos sugerindo que até 50% das perdas gestacionais são devidas a anormalidades cromossômicas (SCIENCE OF BIOGENETICS, 2023). 

Dito isso, a seleção de embriões euploides em procedimentos de fertilização in vitro (FIV) é determinante para o sucesso dos tratamento, porém é baseada em técnicas invasivas, como o Teste Genético Pré-implantacional para Aneuploidia (PGT-A) que é um procedimento relativamente complicado porque requer uma biópsia do embrião, durante a qual danos mínimos ao embrião devem ser garantidos (YANG et al., 2024). O PGT-A, tem um valor prognóstico significativo na tecnologia de reprodução assistida (TORRES, 2009), além da espera prolongada pelos resultados dos exames, que podem demorar até 45 dias úteis depois de enviados para os laboratórios (Dasa Genômica, 2023).

Diante da crescente demanda por técnicas mais acessíveis e menos invasivas na medicina reprodutiva, se torna necessário criar novas abordagens que ofereçam alternativas eficazes aos métodos tradicionais, o que é exatamente o que esse projeto visa fazer.


\section{Objetivos}

\subsection{Objetivos Gerais}
Desenvolver um modelo de Inteligência Artificial capaz de identificar padrões em dados morfocinéticos de embriões, obtidos pela análise feita por embriologistas, que predigam a porcentagem de euploidia evidenciando a saúde genética do embrião, proporcionando uma alternativa menos invasiva e mais acessível para a seleção de embriões em tratamentos de fertilização in vitro.

\subsection{Objetivos Específicos}
\begin{itemize}
    \item \textbf{OE1}: Identificação de Parâmetros em Embriões
    \item \textbf{OE2}: Pré-Processamento de Dados
    \item \textbf{OE3}: Treinamento e Ajuste de Modelos de Machine Learning para Predição de Euploidia
    \item \textbf{OE4}: Avaliação de Confiabilidade do Modelo
    \item \textbf{OE5}: Desenvolvimento de Interface para Predições
\end{itemize}

\section{Metodologia}
A metodologia do projeto está dividida em 5 fases, onde cada uma visa resolver um Objetivo Específico do projeto. Cada fase é composta pelas seguintes atividades:

\begin{itemize}
    \item \textbf{Fase 1 (OE1) - Identificação de Parâmetros em Embriões}
    \begin{itemize}
        \item \textbf{Atividade 1 (A1)}: Analisar as variáveis disponíveis e sua organização;
        \item \textbf{Atividade 2 (A2)}: Detectar e analisar valores que não estão na tabela, que possam afetar a correlação entre os parâmetros e a porcentagem de euploidia;
        \item \textbf{Atividade 3 (A3)}: Calcular a correlação entre cada parâmetro e o impacto de euploidia.
    \end{itemize}
    
    \item \textbf{Fase 2 (OE2) - Pré-Processamento de Dados}
    \begin{itemize}
        \item \textbf{Atividade 4 (A4)}: Realizar a limpeza dos dados, removendo inconsistências, duplicidades e valores ausentes para garantir a consistência dos dados de entrada;
        \item \textbf{Atividade 5 (A5)}: Normalizar e transformar as variáveis para padronizar os formatos e escalas dos dados de entrada.
        \item \textbf{Atividade 6 (A6)}: Definir e atribuir pesos específicos a cada parâmetro, baseando-se em sua influência na ploidia do embrião.
        \item \textbf{Atividade 7 (A7)}: Gerar um gráfico para visualizar os pesos específicos entre os parâmetros e a euploidia.
    \end{itemize}
    
    \item \textbf{Fase 3 (OE3) - Treinamento e Ajuste de Modelo de Machine Learning para Predição de Euploidia}
    \begin{itemize}
        \item \textbf{Atividade 8 (A8)}: Separar o conjunto de dados em conjuntos de treinamento, validação e teste, fazendo uma distribuição dos dados.
        \item \textbf{Atividade 9 (A9)}: Selecionar e configurar modelos pré-treinados ou customizados para a tarefa de predição de euploidia, ajustando o modelo de acordo com o tipo de dados e o objetivo.
        \item \textbf{Atividade 10 (A10)}: Configurar a busca de hiperparâmetros para explorar os melhores valores e otimizar o modelo, aprimorando sua precisão e minimizando erros.
        \item \textbf{Atividade 11 (A11)}: Utilizar um conjunto de validação para monitorar o desempenho inicial e registrar métricas.
    \end{itemize}
    
    \item \textbf{Fase 4 (OE4) - Avaliação de Confiabilidade do Modelo}
    \begin{itemize}
        \item \textbf{Atividade 12 (A12)}: Definir as métricas de avaliação mais adequadas para medir a confiança do modelo, como Accuracy, Precision, Recall, F1-Score, e ROC-AUC (Receiver Operating Characteristic - Area Under Curve), de acordo com a natureza do problema de classificação.
        \item \textbf{Atividade 13 (A13)}: Construir uma matriz de confusão para avaliar o desempenho do modelo em prever corretamente casos de euploidia e aneuploidia.
        \item \textbf{Atividade 14 (A14)}: Gerar a curva ROC para o modelo, o que indica a capacidade do modelo de distinguir entre classes (euploide vs. aneuploide).
        \item \textbf{Atividade 15 (A15)}: Realizar testes de robustez e análise de erros para identificar possíveis áreas de melhoria.
    \end{itemize}
    
    \item \textbf{Fase 5 (OE5) - Desenvolvimento de Interface para Predições}
    \begin{itemize}
        \item \textbf{Atividade 16 (A16)}: Prototipar uma interface amigável para exibir as predições de euploidia para o usuário final (médicos).
        \item \textbf{Atividade 17 (A17)}: Integrar a interface com o modelo de Machine Learning, garantindo respostas em tempo real.
        \item \textbf{Atividade 18 (A18)}: Realizar testes de usabilidade para otimizar a experiência do usuário médico e assegurar fácil interpretação dos resultados.
    \end{itemize}
\end{itemize}


\subsection{Organização da Pesquisa}
A Figura 1 mostra como a pesquisa deste trabalho irá ocorrer, com suas atividades de entrada e saída. Essa imagem foi criada para melhorar o entendimento do leitor.

\section{Composição e estrutura do trabalho}
Este trabalho foi organizado da seguinte maneira:

\textbf{Capítulo 2}: intitulado "Referencial Teórico", foi abordado os principais conceitos que apoiam a contextualização deste trabalho;\\

\textbf{Capítulo 3}: intitulado "Metodologia", detalhou seção 1.5. Será descrito os procedimentos e métodos utilizados na pesquisa indicando um planejamento de trabalho, as atividades realizadas e os resultados esperados;\\

\textbf{Capítulo 4}: intitulado "Planejamento", apresenta a previsão de término de cada atividade e fase propostas;\\

\textbf{Capítulo 5}: intitulado "Execução e Resultados Preliminares", apresenta o que foi concluído durante o período de escrita deste TCC;\\

\textbf{Capítulo 6}: intitulado "Considerações e Trabalhos Futuros", foi retomada todas as atividades e produtos obtidos até o momento da escrita do capítulo.
