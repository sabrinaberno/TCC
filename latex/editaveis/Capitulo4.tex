\chapter[Metodologia]{Execução da Pesquisa}

\section{Fase 1: Análise e Preparação de Dados}
\subsection{OE1 - Expansão, Processamento e Análise de Dados para Predição de Ploidia}
\subsubsection{Atividade 1 (A1): Análise, Revisão, Seleção e Limpeza de Variáveis para Predição de Euploidia}
O desenvolvimento dessa atividade teve início com a elaboração do referencial teórico, abordando os conceitos fundamentais relacionados ao aprendizado supervisionado e às tecnologias aplicadas no contexto da fertilização in vitro, além de nos aprofundarmos em pesquisas relacionadas a área da medicina reprodutiva. A pesquisa teórica foi direcionada ao entendimento dos padrões morfocinéticos de embriões capturados por sistemas de Time-Lapse e de como essas informações podem ser utilizadas para prever a porcentagem de euploidia, um indicador crítico da saúde genética do embrião.

Os dados utilizados neste estudo foram obtidos por meio do especialista, \citeonline{ramalho2024}, e de suas pacientes, que autorizaram o uso dessas informações para fins de pesquisa, contratos esses que estão na sessão de Anexo. As informações foram coletadas na clínica Bruno Ramalho Reprodução Humana, bem como na clínica Genesis, onde o \citeonline{ramalho2024} também atua.

A primeira etapa dessa atividade consiste em verificar a pertinência das variáveis já existentes na \textbf{“Planilha de dados dos embriões 1”}, que se encontra em Anexos, analisando seu impacto no desempenho do modelo preditivo. Essa análise também busca identificar se há necessidade de introduzir novas variáveis que possam melhorar a precisão do modelo ou excluir aquelas que se mostram irrelevantes.

\subsubsection{Idade}
As informações trazidas pelo Fertility and Ageing corroboram a relevância de incluir variáveis relacionadas à idade materna, pois estudos apontam que o aumento da aneuploidia em embriões está diretamente associado ao envelhecimento materno. O estudo de \citeonline{yuan2023} também aponta que a taxa de euploidia dos embriões está correlacionada com a idade feminina. À medida que a idade avança, há um declínio no número total e na qualidade dos ovócitos, um fator crítico para a fecundidade reduzida observada após os 35 anos. Além disso, a resposta à estimulação ovariana e os níveis de FSH (hormônio folículo-estimulante) emergem como possíveis variáveis preditivas relevantes. Estudos indicam que a idade materna exerce maior influência sobre a qualidade embrionária do que os níveis de FSH isoladamente, reforçando que o impacto na fertilidade está mais relacionado à qualidade do oócito do que à sua quantidade \cite{eshre2005}.

\subsubsection{t2, t3, t4, t5, t8, s2, cc2, tSC, tSB, tB, cc3 (t5-t3), s3 (t8-t5), t5-t2,  tSC-t8 e tB-tSB}
Os \textbf{sistemas de Time-Lapse} ajudam a identificar marcadores morfocinéticos, que mostram como as células se dividem durante o desenvolvimento do embrião. Esses marcadores, junto com características físicas tradicionais, são fundamentais para selecionar o embrião mais adequado para a transferência \cite{souzarebeca2022}. O desenvolvimento embrionário é um processo dinâmico, com mudanças perceptíveis em um curto período \cite{cruz2012}. Estudos detalhados sobre o ritmo das divisões celulares, assim como características como tamanho e organização das células, demonstram que o tempo necessário para atingir certos estágios de desenvolvimento está diretamente relacionado ao potencial de implantação do embrião \cite{souzarebeca2022}.

Embriões que se dividem muito rapidamente apresentam menor chance de implantação quando comparados aos que seguem um ciclo celular dentro do intervalo considerado normal  \cite{cruz2012}. Isso ocorre porque alterações no tempo de compactação inicial, na formação da blástula e na progressão até o estágio de blastocisto completo estão associadas a uma maior probabilidade de aneuploidias \cite{cruz2012}. O projeto “Timing of cell division in human cleavage-stage embryos is linked with blastocyst formation and quality” de \citeonline{cruz2012} utilizou o sistema Time-Lapse para identificar marcadores morfocinéticos e monitorar com precisão os tempos das divisões celulares durante o desenvolvimento embrionário. Os tempos considerados ótimos para previsões de desenvolvimento embrionário foram: \textbf{t2 (24,3–27,9 horas), t3 (35,4–40,3 horas), t5 (48,8–56,6 horas), s2 (<0,76 horas) e cc2 (<11,9 horas)}. A explicação detalhada dessas variáveis está disponível no \textbf{Apêndice A}. 

Especificamente, no nível morfológico, \textbf{t5} destacou-se como o indicador mais relevante do potencial de implantação \cite{cruz2012}. Observa-se que a capacidade de diferenciar embriões viáveis daqueles não viáveis melhora significativamente quando os critérios se baseiam em eventos de divisão celular mais tardios \cite{cruz2012}. Embriões com t5 entre 48,8 e 56,6 horas demonstram não apenas um maior potencial de implantação, mas também uma maior propensão a se desenvolverem em blastocistos de morfologia superior \cite{cruz2012}.

Ao criar um modelo de IA para a previsão de euploidia, é importante que considere as variáveis \textbf{t4, t8, tSC, tSB, tB, cc3 (t5 - t3) e s3 (t8 - t5)}. Essas métricas são fundamentais para o desenvolvimento embrionário, conforme evidenciado no estudo que comprovou a efetividade da IA em detectar embriões viáveis com uma precisão de 70\% \cite{borges2019}. A incorporação dessas variáveis em um modelo de IA é crucial, pois possibilita a captura das sutilezas do desenvolvimento embrionário, que, de acordo com a pesquisa, são melhoradas através da análise automatizada fundamentada em IA. Isso se torna especialmente relevante pois os sistemas de time-lapse disponibilizam informações detalhadas e contínuas, que podem ser combinadas para detectar padrões relacionados à euploidia \cite{borges2019}.

A avaliação de fatores morfocinéticos, como o tempo necessário para clivagem e a extensão das fases subsequentes, tem sido alvo de pesquisa para antecipar a probabilidade de implantação embrionária. Por exemplo, um estudo publicado pelo Instituto Sapientiae por Edson Borges Jr. destacou a importância desses parâmetros na predição do potencial de implantação embrionária. Apesar dessa pesquisa não tratar especificamente os intervalos \textbf{t5-t2, tSC-t8 e tB-tSB}, ela sugere que a avaliação de intervalos de tempo entre eventos específicos no desenvolvimento embrionário pode oferecer percepções valiosas sobre a qualidade e a capacidade de desenvolvimento dos embriões \cite{borges2022}.

\subsubsection{Estágio e Morfo}
Modelos de IA têm sido empregados na avaliação de embriões produzidos até o quinto dia ou mais, com o objetivo de aprimorar a escolha com base em informações objetivas e de alta exatidão \cite{lassen2022}. Esses modelos priorizam a análise dos estágios "\textbf{Dia 5+}", ou seja, aqueles que atingem o estágio de blastocisto no quinto dia ou mais tarde, devido à maior disponibilidade de dados morfológicos e dinâmicos do desenvolvimento embrionário \cite{lassen2022}. Conforme o Dr. Bruno Ramalho, esses elementos não afetam diretamente a ploidia dos embriões, mas sim o seu desenvolvimento, ressaltando a relevância de reconhecer padrões que possam antecipar a viabilidade embrionária.

De acordo com os critérios definidos por \citeonline{gardner1999}, com base na morfologia do embrião se tem a categorização dos blastocistos, um fator determinante para o potencial de implantação e a qualidade embrionária \cite{capalbo2014}. Os blastocistos são agrupados em quatro categorias principais, considerando tanto a massa celular interna (ICM, Inner Cell Mass) quanto o TE (trophectoderma):
\begin{itemize}
  \item \textbf{Grupo 1 (Excelente)}: Blastocistos com classificação \textbf{≥3AA}.Blastocistos altamente desenvolvidos com massa celular interna densa e trophectoderma bem organizado \cite{capalbo2014}.
  \item \textbf{Grupo 2 (Bom)}: Blastocistos com classificação \textbf{3, 4, 5 ou 6} e com notas \textbf{AB ou BA}. Apresentam características boas, mas menos consistentes em relação ao grupo excelente \cite{capalbo2014}.
  \item \textbf{Grupo 3 (Médio)}: Blastocistos com classificação \textbf{3, 4, 5 ou 6} e notas \textbf{BB, AC ou CA}. Qualidade moderada com irregularidades tanto na ICM quanto no TE \cite{capalbo2014}.
  \item \textbf{Grupo 4 (Ruim)}: Blastocistos com classificação \textbf{≤3BB}. Blastocistos de menor qualidade, com poucas células organizadas na ICM e TE menos coeso \cite{capalbo2014}.
\end{itemize}
O estudo de \citeonline{capalbo2014} enfatizou a relação entre a morfologia padrão dos blastocistos, a euploidia e as taxas de implantação. Blastocistos de excelente morfologia, particularmente os biopsiados no dia 5, mostraram uma maior probabilidade de serem euploides e apresentaram taxas de implantação superiores.

\subsubsection{KIDScore™}
O KIDScore™ é um algoritmo baseado em IA, aplicado à análise de imagens em sistemas Time-Lapse, tem se mostrado uma ferramenta importante na avaliação de embriões durante os tratamentos de reprodução assistida \cite{kato2021}. O Algoritmo combina variáveis morfocinéticas e parâmetros de desenvolvimento embrionário para fornecer uma pontuação que auxilia na seleção de embriões com maior potencial de implantação e viabilidade genética \cite{gazzo2020}. A pontuação vai de 0 a 10. Pontuações baixas, entre 0 e 3, indicam embriões de qualidade inferior, com baixo potencial de implantação. Pontuações médias, de 4 a 6, correspondem a embriões de qualidade moderada, com um potencial razoável de implantação. Já pontuações altas, de 7 a 10, representam embriões de alta qualidade, com grande potencial de implantação \cite{gazzo2020}. \citeonline{kato2021} cita que o modelo apresentou uma alta precisão na previsão de resultados de gravidez, sendo especialmente útil tanto em pacientes com idade materna avançada quanto em pacientes mais jovens. Por fim, \citeonline{gazzo2020} informaram que o uso do algoritmo no processo de seleção embrionária levou a um aumento expressivo nas taxas de implantação após a transferência de embriões congelados (FET). Então o modelo quando combinada com informações sobre os tempos de divisão celular, sendo \textbf{t2, t3, t5, s2 e cc2} descritos por \citeonline{cruz2012}, possibilita um exame mais completo do crescimento embrionário, melhorando a acurácia na seleção de embriões com maior probabilidade de êxito em tratamentos de FIV. 

\subsubsection{KIDScore™}
Para a elaboração do nosso modelo de previsão de euploidia, optamos por empregar a coluna de \textbf{Plodia}, que proporciona uma categorização minuciosa dos embriões em diversas categorias de euploidia. As categorizações contidas nesta coluna são: Aneuplóide complexo, Aneuploide/Triploide XXX, Caótico, Haploide, Mosaico de alto grau, Mosaico de baixo grau e Normal/Euplóide. Em um dos diálogos com o especialista Dr. Bruno Ramalho, ele nos deu uma orientação clínica crucial sobre como categorizar o mosaico. De acordo com o \citeonline{ramalho2024}, em termos clínicos, geralmente considera-se os embriões com Caótico, Haplóide e Mosaico de alto grau como Aneuploides, enquanto os com mosaico de baixo grau como Euploides. Diante disso, optamos por reorganizar os valores na tabela de dados, agrupando as seguintes categorias sob o termo \textbf{Aneuploide}: Aneuploide complexo, Aneuploide/Triploide XXX, Caótico, Haploide e Mosaico de alta complexidade. Em contrapartida, os embriões categorizados como Mosaico de baixo grau e Normal/Euploide serão reunidos na categoria \textbf{Euploide}.

\subsubsection{Limpeza dos Dados}
Ao lidar com os dados ausentes, utilizamos o método de Análise de Casos Completos (ACC), que envolve a remoção de observações que possuem pelo menos um valor ausente \cite{camargos2011}. Este procedimento é frequentemente empregado quando o número de dados ausentes é reduzido, assegurando que a remoção de algumas observações não interfira de forma significativa na análise estatística e preserva a consistência do modelo \cite{camargos2011}. Em nosso cenário, das 85 linhas de dados disponíveis, apenas 2 apresentavam valores ausentes. As células sem dados estavam em preto na \textbf{"Planilha de dados dos embriões 2"}. 




