\chapter[Considerações e Trabalhos Futuros]{Considerações e Trabalhos Futuros}

Até o presente momento, a pesquisa evoluiu atingindo marcos importantes na estruturação do modelo de Aprendiziado de Maquina para a predição da euploidia em embriões a partir de dados morfocinéticos. O objetivo dessa primeira etapa foi realizar o tratamento dos dados, além de também entender como cada variável da nossa tabela de dados se influenciavam e na ploidia. 

\section{Objetivos Cumpridos e Fases Concluídas}
\subsection{Fase 1: Análise e Preparação de Dados}
A realização da "Análise, Revisão, Seleção e Limpeza de Variáveis", assim, se identificou variáveis relevantes para a previsão de euploidia, com base em revisões bibliográficas e critérios científicos.

A "Normalização dos Dados", com a aplicação do método Z-Score, permitiu equalizar a escala das variáveis, garantindo maior precisão na análise.

Identificação da correlação entre variáveis, com o coeficiente de Spearman, se destacou relações significativas entre variáveis como a idade materna e os tempos de desenvolvimento embrionário (tB-tSB e CC3) na previsão da euploidia.

E por fim, a divisão e aumento de dados, implementadando a separação em conjuntos de treinamento, validação e teste, complementada por técnicas de aumento de dados para preservar padrões estatísticos e evitar overfitting.

Dessa maneira, finalizamos a Fase 1, que inclui o objetivo específico 1, de "Expansão, Processamento e Análise de Dados para Predição de Ploidia". 

\section{Descobertas e Resultados Preliminares}

A \textbf{idade materna}, com correlação negativa de \textbf{-0,50}, é o fator mais relevante na qualidade genética do embrião, consequentemente sendo o fato que mais afeta negativamente a ploidia. Esse dado reforça a importância de considerar a idade como parâmetro essencial em tratamentos de fertilização in vitro.

Variáveis como o estágio de desenvolvimento e o tempo \textbf{t5} apresentam correlações moderadas e negativas (-0,24), sugerindo que atrasos em etapas particulares do ciclo embrionário podem afetar adversamente a qualidade genética.

Sobre os \textbf{tempos de desenvolvimento embrionário}, a diferenças entre \textbf{tB-tSB (-0,28) e CC3 (-0,28)} indicam que a sincronização das divisões celulares é fundamental para a formação de embriões euploides. Além de que apresentaram correlações moderadas com a ploidia, sugerindo que o sincronismo no desenvolvimento celular é um elemento-chave na formação de embriões saudáveis.

Com a normalização, o ajuste das escalas das variáveis permitiu maior uniformidade e comparabilidade entre elas, facilitando a identificação de padrões no modelo preditivo.

\section{Considerações Finais}

A progressão do projeto até este ponto demonstra avanços na compreensão dos padrões morfocinéticos que influenciam a euploidia. A conclusão da Fase 1 permitiu não apenas a limpeza e preparação dos dados, mas também a identificação de relações estatisticamente relevantes entre as variáveis.

Com os dados devidamente estruturados, o próximo passo será o treinamento do modelo, buscando refinar a capacidade da IA de oferecer uma alternativa menos invasiva e mais acessível para a seleção de embriões em tratamentos de fertilização in vitro. A seleção criteriosa das variáveis, embasada em literatura relevante, assegurou que o modelo fosse construído sobre parâmetros cientificamente validados. Além disso, a limpeza e normalização dos dados eliminaram vieses e prepararam um conjunto de dados consistente, refletindo uma metodologia rigorosa focada na confiabilidade dos resultados.

Dessa forma, os resultados obtidos até o momento reforçam o potencial do modelo na predição da euploidia embrionária, contribuindo para avanços na medicina reprodutiva. Trabalhos futuros devem focar no treinamento e validação do modelo.