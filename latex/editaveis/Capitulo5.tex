\chapter[Conclusões e Trabalhos Futuros]{Conclusões e Trabalhos Futuros}

Até o presente momento, a pesquisa evoluiu atingindo marcos importantes na estruturação do modelo de Aprendiziado de Maquina para a predição da euploidia em embriões a partir de dados morfocinéticos. O objetivo dessa primeira etapa foi realizar o tratamento dos dados, além de também entender como cada variável da nossa tabela de dados se influenciavam e na ploidia. 

\section{Objetivos Cumpridos e Fases Concluídas}
\subsection{Fase 1: Análise e Preparação de Dados}
A realização da "Análise, Revisão, Seleção e Limpeza de Variáveis" identificou variáveis relevantes para a previsão de euploidia, com base em revisões bibliográficas e critérios científicos.

A "Normalização dos Dados" permitiu equalizar a escala das variáveis, garantindo maior precisão na análise.

Identificação da correlação entre variáveis, com o coeficiente de Spearman, se destacou relações significativas entre variáveis como a idade materna e os tempos de desenvolvimento embrionário (tB-tSB e CC3) na previsão da euploidia.

E por fim, a divisão e aumento de dados, implementadando a separação em conjuntos de treinamento, validação e teste, complementada por técnicas de aumento de dados para preservar padrões estatísticos e evitar overfitting.

Dessa maneira, finalizamos a Fase 1, que inclui o objetivo específico 1, de "Expansão, Processamento e Análise de Dados para Predição de Ploidia". 

\subsubsection{Descobertas e Resultados Preliminares da Fase 1}

A \textbf{idade materna}, com correlação negativa de \textbf{-0,50}, é o fator mais relevante na qualidade genética do embrião, consequentemente sendo o fato que mais afeta negativamente a ploidia. Esse dado reforça a importância de considerar a idade como parâmetro essencial em tratamentos de fertilização in vitro.

Variáveis como o estágio de desenvolvimento e o tempo \textbf{t5} apresentam correlações moderadas e negativas (-0,24), sugerindo que atrasos em etapas particulares do ciclo embrionário podem afetar adversamente a qualidade genética.

Sobre os \textbf{tempos de desenvolvimento embrionário}, a diferenças entre \textbf{tB-tSB (-0,28) e CC3 (-0,28)} indicam que a sincronização das divisões celulares é fundamental para a formação de embriões euploides. Além de que apresentaram correlações moderadas com a ploidia, sugerindo que o sincronismo no desenvolvimento celular é um elemento-chave na formação de embriões saudáveis.

Apesar de algumas variáveis apresentarem uma correlação direta fraca ou inexistente com a ploidia — como é o caso do \textit{Morfo}, \textit{tSC}, \textit{tSB}, \textit{cc2} e o intervalo \textit{tSC-t8}, que demonstra um efeito praticamente nulo — é importante destacar que essas variáveis mantêm entre si relações estatísticas relevantes, capazes de influenciar, de forma indireta, o comportamento da ploidia. 

As inter-relações entre os parâmetros temporais do desenvolvimento embrionário constroem um cenário dinâmico e multifatorial, no qual o impacto de uma variável sobre a ploidia pode não ser imediato ou isolado, mas sim mediado por sua influência em outras etapas do desenvolvimento. Por exemplo, mesmo que o \textit{cc2} (\textit{t3-t2}) não apresente uma relação significativa direta com a ploidia, ele se conecta a outros parâmetros, como o \textit{t3} e o \textit{cc3}, sendo que este último já demonstra uma correlação moderada e negativa com a ploidia (-0.281).

Da mesma forma, variáveis como o \textit{tSC} e o intervalo \textit{tSC-t8}, apesar de mostrarem uma correlação direta praticamente nula com a ploidia, estão integradas a outros intervalos temporais que, por sua vez, têm potencial de afetar o desfecho genético do embrião. Isso evidencia que o sistema de desenvolvimento embrionário funciona de maneira interdependente, onde alterações em um ponto específico podem reverberar ao longo do processo, impactando de forma indireta variáveis de interesse clínico, como a ploidia.

Portanto, ao considerar a exclusão de variáveis do modelo ou da análise, é fundamental adotar uma abordagem sistêmica, reconhecendo que a ausência de uma relação direta com a ploidia não implica na irrelevância da variável dentro do contexto geral do desenvolvimento embrionário. Desconsiderar essas variáveis pode levar a interpretações incompletas ou equivocadas do comportamento biológico em estudo.

\subsection{Fase 2:  Desenvolvimento e Avaliação do Modelo}
A segunda fase do projeto teve como principal objetivo transformar o conhecimento gerado na etapa anterior em uma solução funcional de predição de euploidia, baseada em inteligência artificial, acessível a médicos e profissionais da saúde reprodutiva. A partir das análises realizadas na Fase 1, que revelaram correlações importantes entre variáveis morfocinéticas e a ploidia embrionária, foi possível estruturar um modelo preditivo robusto e iniciar o desenvolvimento de uma interface prática e responsiva.

\subsubsection{Resultados do Modelo Preditivo}
Com base nas variáveis analisadas e selecionadas, desenvolveu-se uma Rede Neural Multicamadas (MLP) capaz de prever, com base nos dados de um embrião, a probabilidade de ele ser euploide. O modelo alcançou uma acurácia de 88,2\% e uma AUC de 0,944, indicando alta capacidade discriminativa entre embriões euploides e aneuploides. Além disso, a utilização do método LIME permitiu interpretar e visualizar quais variáveis mais contribuíram para cada predição individual, promovendo transparência e confiabilidade, essenciais no contexto médico.

Esses resultados foram viabilizados graças à estrutura sólida construída na Fase 1, que realizou um trabalho criterioso de normalização, balanceamento, análise estatística e aumento de dados. As descobertas de correlações significativas (como a influência da idade materna, tB-tSB, CC3, e tempo de clivagem) foram cruciais para orientar a arquitetura do modelo e sua performance atual.

\subsubsection{Benefícios Práticos para o Uso Clínico}
A integração do modelo a uma aplicação web proporciona acesso prático aos resultados preditivos em tempo real. A interface, desenvolvida com Next.js, TypeScript, TailwindCSS e conectada via FastAPI, permite que o médico:
\begin{itemize}
    \item Faça o upload de uma planilha com dados morfocinéticos;

    \item Visualize os resultados da predição de forma imediata e intuitiva;

    \item Receba mensagens claras de erro em caso de problemas com o arquivo enviado;

    \item Entenda quais características impactaram na predição de cada embrião.
\end{itemize}

Esses elementos aumentam significativamente a agilidade e confiabilidade na tomada de decisão clínica, reduzindo o tempo necessário para análises manuais e possibilitando um suporte adicional na escolha de embriões viáveis para implantação.

A interface foi projetada com base em um MVP funcional, garantindo simplicidade, acessibilidade e foco nos recursos mais críticos. A solução foi continuamente testada, validada e disponibilizada via deploy contínuo na plataforma Vercel, garantindo sua estabilidade e reprodutibilidade.

\section{Considerações Finais}

A execução deste trabalho permitiu o desenvolvimento de uma solução inovadora e menos invasiva para a predição da euploidia em embriões humanos, integrando conhecimentos de engenharia de software, aprendizado de máquina e medicina reprodutiva.

Na Fase 1, se realizou a análise e preparação dos dados, etapa crucial que envolveu a seleção criteriosa das variáveis morfocinéticas e morfológicas relevantes, bem como a limpeza, normalização e expansão da base de dados. A aplicação de testes estatísticos, como o coeficiente de Spearman, permitiu identificar quais parâmetros tinham maior ou menor influência na ploidia dos embriões. Além disso, a normalização dos dados utilizando o z-score possibilitou o balanceamento das escalas entre as variáveis, garantindo que o modelo pudesse aprender de forma equitativa e evitar viéses relacionados à magnitude dos dados.

Essa base sólida construída na Fase 1 foi determinante para o desempenho da Fase 2, que consistiu no desenvolvimento, treinamento e avaliação dos modelos de machine learning. O conhecimento obtido anteriormente possibilitou a construção de algoritmos mais precisos, já que os dados de entrada estavam devidamente preparados e os parâmetros relevantes estavam bem compreendidos. A integração entre as duas fases é evidente: sem a etapa rigorosa de análise e preparação, o modelo não teria alcançado o desempenho elevado que foi observado na classificação da ploidia.

Além do desenvolvimento técnico, o projeto também contemplou a criação de uma interface básica voltada ao público médico, permitindo a visualização direta das predições realizadas pela IA, com foco na aplicabilidade prática da solução em contextos clínicos.

De forma geral, este projeto demonstrou que é possível utilizar inteligência artificial para fornecer suporte à decisão médica na escolha de embriões com maior potencial de implantação, reduzindo a dependência de métodos invasivos como o PGT-A. A abordagem proposta mostra-se promissora tanto por sua viabilidade técnica quanto por seu potencial de impacto social, ao tornar os tratamentos de fertilização mais acessíveis, rápidos e seguros.