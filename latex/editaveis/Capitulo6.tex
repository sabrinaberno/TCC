\chapter[Trabalhos futuros]{Trabalhos futuros}

% \begin{itemize}
%     \item \textbf{A1} - Análise, revisão, seleção e limpeza de variáveis para predição de euploidia.
%     \item \textbf{A2} - Normalização dos dados para otimização.
%     \item \textbf{A3} - Identificação da correlação e atribuição de pesos aos parâmetros na previsão da ploidia do embrião.
%     \item \textbf{A4} - Divisão dos dados e aplicação de \textit{Data Augmentation}.
%     \item \textbf{Apr. TCC1}. - Apresentação do TCC1
%     \item \textbf{Revisão TCC1} - Ajustes no TCC1 solicitados pela banca e pelo orientador
%     \item \textbf{A5} - Desenvolvimento e Treinamento do Modelo de Machine Learning
%     \item \textbf{A6} - Utilizar métricas adequadas para medir o desempenho do modelo 
%     \item \textbf{A7} - Avaliação do Desempenho do Modelo na Predição por meio da Matriz de Confusão e Curva ROC
%     \item \textbf{A8} - Prototipar uma interface
%     \item \textbf{Apr. TCC2} - Apresentação do TCC2
% \end{itemize}

% \renewcommand{\arraystretch}{1.2}
% \begin{table}[ht]
%     \centering
%     \caption{Cronograma de atividades do TCC}
%     \begin{tabular}{|l|c|c|c|c|c|c|c|c|}
%     \hline
%     \cellcolor[HTML]{008940}\textbf{Atividades} & \cellcolor[HTML]{008940}\textbf{Dez} & \cellcolor[HTML]{008940}\textbf{Jan} & \cellcolor[HTML]{008940}\textbf{Fev} & \cellcolor[HTML]{008940}\textbf{Mar} & \cellcolor[HTML]{008940}\textbf{Abr} & \cellcolor[HTML]{008940}\textbf{Mai} & \cellcolor[HTML]{008940}\textbf{Jun} & \cellcolor[HTML]{008940}\textbf{Jul}\\ \hline
%     A1 & X & X &   &   &   &   &   &  \\ \hline
%     A2 &   & X &   &   &   &   &   &  \\ \hline
%     A3 &   & X &   &   &   &   &   &  \\ \hline
%     A4 &   & X &   &   &   &   &   &  \\ \hline
%     Apr. TCC1 &   & X &   &   &   &   &   &  \\ \hline
%     Revisão TCC1 &   &   &   & X &   &   &   &  \\ \hline
%     A5 &   &   &   & X & X &   &   &  \\ \hline
%     A6 &   &   &   & X & X &   &   &  \\ \hline
%     A7 &   &   &   &   &   & X &   &  \\ \hline
%     A8 &   &   &   &   &   &   & X & X \\ \hline
%     Apr. TCC2 &   &   &   &   &   &   &   & X \\ \hline
%     \end{tabular}
%     \label{tab:cronograma}
% \end{table}

A aplicação desenvolvida até o momento apresenta uma solução funcional e eficaz para a predição de euploidia em embriões a partir de dados morfocinéticos, integrando um modelo de inteligência artificial com uma interface web acessível. No entanto, existem diversas oportunidades de evolução que podem aprimorar ainda mais a performance técnica, a usabilidade e a aplicabilidade clínica da plataforma.

Atualmente, todas as validações de entrada — como o número de colunas na planilha, a conformidade com o formato esperado e a detecção de erros estruturais — são realizadas diretamente na aplicação web, antes do envio dos dados à inteligência artificial. A primeira melhoria prevista consiste em transferir essas responsabilidades para o próprio modelo de IA. Isso permitiria que o sistema, de forma autônoma, identificasse e tratasse inconformidades nos dados de entrada, tornando o processo mais robusto, dinâmico e menos dependente de verificações manuais na interface.

Outra melhoria futura relevante é a implementação de um \textbf{sistema de armazenamento e gerenciamento de histórico clínico individualizado}. A ideia é que cada paciente (ou embrião) possa ter seus dados armazenados de forma segura e identificável, permitindo ao médico acessar rapidamente o histórico completo de embriões avaliados, suas respectivas predições de euploidia e demais informações clínicas relevantes. Isso eliminaria a necessidade de reenviar planilhas repetidas vezes e promoveria um controle mais preciso e organizado dos dados clínicos de cada paciente.

Também se propõe a\textbf{integração da plataforma com os sistemas já utilizados por clínicas de fertilização e consultórios médicos}. Ao permitir a interoperabilidade com softwares médicos amplamente adotados em consultórios e clínicas de fertilização, espera-se otimizar o fluxo de trabalho e reduzir barreiras de adoção por parte dos profissionais da área da saúde.

Adicionalmente, propõe-se o desenvolvimento de um \textbf{Painel Analítico Interativo para Médicos}, concebido como um dashboard dinâmico que permita a visualização de gráficos de distribuição dos embriões por paciente, histórico de predições com filtros por período, idade materna, entre outros critérios, e a análise de tendências clínicas com base nos dados preditivos acumulados. Tal painel tem o potencial de transformar a plataforma em uma \textbf{ferramenta robusta de apoio à decisão baseada em dados (data-driven)}, oferecendo aos médicos uma visão ampla e estratégica das informações clínicas, indo além da análise pontual e possibilitando o acompanhamento contínuo de padrões relevantes no contexto da medicina reprodutiva.

Esses aprimoramentos visam não apenas o avanço técnico da plataforma, mas também sua incorporação efetiva na rotina médica, contribuindo para decisões mais rápidas, precisas e seguras no contexto da medicina reprodutiva. O projeto, portanto, abre caminhos promissores para a consolidação de tecnologias de IA como suporte real à prática clínica, com foco em acessibilidade, eficiência e impacto positivo na jornada do paciente. Todas essas melhorias estão alinhadas ao objetivo geral do projeto: desenvolver uma abordagem baseada em inteligência artificial para identificar padrões em dados morfocinéticos de embriões, obtidos por meio do Time-Lapse System, capaz de predizer a porcentagem de euploidia, proporcionando uma solução mais eficaz e menos invasiva em comparação ao PGT-A.