\begin{resumo}[Abstract]
 \begin{otherlanguage*}{english}
    In vitro fertilization is widely used by couples facing reproductive difficulties or those wishing to postpone pregnancy, but its success depends on several factors, with the selection of viable embryos being one of the most critical. Choosing the correct embryo significantly increases the chances of successful implantation, reduces the risk of miscarriage, and improves healthy birth rates. Preimplantation Genetic Testing for Aneuploidy is the most commonly used method, but in addition to its high cost, it has diagnostic precision limitations, which can lead to the transfer of non-viable embryos or exclusion of viable ones. With the advancement of technologies such as the Time-Lapse System, which captures continuous images of embryonic development, the possibility arises of predicting euploidy using morphokinetic data in a less invasive and more effective manner. The research adopts a quantitative and experimental approach, focusing on the analysis of numerical data related to the morphokinetic patterns of embryos for the development of predictive AI. This work develops a Machine Learning-based approach to identify patterns in morphokinetic data and predict euploidy, offering an alternative to PGT-A without invasive interventions. After a literature review and correlation analysis, the most relevant variables were identified, with age and tb-t2b showing the highest negative influence. The Machine Learning models were trained and validated using metrics such as accuracy, sensitivity, and specificity to assess their performance in predicting embryo euploidy based on morphokinetic data. The developed model, based on a Multilayer Perceptron (MLP) Artificial Neural Network, achieved an accuracy of 88.2\% and an Area Under the ROC Curve (AUC) of 0.944, indicating a high discriminative ability between euploid and aneuploid embryos. The evaluation through the confusion matrix demonstrated excellent performance in identifying aneuploid embryos, with perfect recall (1.00) for this class and robust recall (0.75) for euploid embryos, without any false positives. The integration of the LIME method made the model's predictions interpretable by identifying which morphokinetic variables most influenced each individual prediction, an essential feature to ensure transparency and reliability in the biomedical context. The developed solution offers a promising tool to complement PGT-A, reducing costs and avoiding invasive interventions, with potential for real-world application in the clinical routine of assisted reproduction.
  \vspace{\onelineskip}
 
  \noindent 
  \textbf{Key-words}: In vitro fertilization. Euploidy. Time-Lapse System. PGT-A. Artificial Intelligence. Machine Learning. Data analysis.
 \end{otherlanguage*}
\end{resumo}
