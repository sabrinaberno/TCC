\begin{resumo}[Abstract]
 \begin{otherlanguage*}{english}
  In vitro fertilization is widely used for couples with reproductive difficulties or those wishing to delay pregnancy, but it still faces significant challenges, particularly in relation to the selection of viable embryos. Approximately 50\% of miscarriages are associated with chromosomal abnormalities, making the selection of euploid embryos essential to increase the success rates of the procedure. Preimplantation Genetic Testing for Aneuploidy is currently the most commonly used method for this selection, but it involves embryo cell biopsy, an invasive procedure that can harm embryo viability, as well as being costly. With the advancement of monitoring technologies such as the Time-Lapse System, which captures continuous images of embryo development, the possibility arises to use morphokinetic data to predict euploidy in a less invasive and more effective manner. This study aims to develop an artificial intelligence-based approach to identify patterns in the morphokinetic data of embryos obtained through the Time-Lapse System, with the goal of predicting the percentage of euploidy and providing an alternative to PGT-A without the need for invasive interventions. The methodology adopted for the study follows a quantitative approach, based on literature review and analysis of correlations between morphokinetic variables of the embryos. From a literature review, it was possible to identify the most relevant variables for analysis, and then correlation analysis between these variables was conducted. The results found in the correlations confirmed the trends observed in the literature, showing that the selected morphokinetic variables have a significant relationship with the prediction of euploidy. Although this initial phase focuses on correlation analysis, it provides a solid foundation for the development of more complex machine learning models that can be applied to predict euploidy more accurately. The process includes data preprocessing, extraction of relevant morphokinetic features, training of AI models, and validation of results with metrics such as accuracy, sensitivity, and specificity. The comparison between the results obtained by AI and PGT-A will serve to assess the effectiveness and accuracy of the proposed approach. It is expected that artificial intelligence, when analyzing morphokinetic data, will present high accuracy in predicting euploidy, exceeding 70\%, with the potential to complement the currently used invasive methods and assist doctors in selecting the best embryo. The application of this solution may result in a significant increase in IVF success rates, while providing a more cost-effective and less harmful approach for embryos. This study has the potential to transform assisted reproduction practices by reducing costs and risks associated with invasive methods, offering a more accessible solution for selecting embryos with the highest probability of successful implantation.

  \vspace{\onelineskip}
 
  \noindent 
  \textbf{Key-words}: In vitro fertilization. Euploidy. Time-Lapse System. PGT-A. Artificial Intelligence. Machine Learning. Data analysis.
 \end{otherlanguage*}
\end{resumo}
