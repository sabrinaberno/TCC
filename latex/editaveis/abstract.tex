\begin{resumo}[Abstract]
 \begin{otherlanguage*}{english}
    In vitro fertilization is widely used by couples facing reproductive difficulties or those wishing to postpone pregnancy, but its success depends on several factors, with the selection of viable embryos being one of the most critical. Choosing the correct embryo significantly increases the chances of successful implantation, reduces the risk of miscarriage, and improves healthy birth rates. Preimplantation Genetic Testing for Aneuploidy is the most commonly used method, but in addition to its high cost, it has diagnostic precision limitations, which can lead to the transfer of non-viable embryos or exclusion of viable ones. With the advancement of technologies such as the Time-Lapse System, which captures continuous images of embryonic development, the possibility arises of predicting euploidy using morphokinetic data in a less invasive and more effective manner. This work proposes a Machine Learning-based approach to identify patterns in morphokinetic data and predict euploidy, offering an alternative to PGT-A without invasive interventions. After a literature review and correlation analysis, the most relevant variables were identified, with age and tb-t2b showing the highest negative influence. The next steps include training Machine Learning models and validating the results using metrics such as accuracy, sensitivity, and specificity. The comparison with PGT-A will assess the approach’s effectiveness. It is expected that Machine Learning analysis of morphokinetic data will achieve high accuracy, above 70\%, complementing invasive methods and assisting in the selection of the best embryo. Applying this solution may significantly increase IVF success rates, while offering a more economical and less harmful alternative. This study has the potential to transform assisted reproduction practices by reducing costs and risks, making embryo selection more accessible and effective.

  \vspace{\onelineskip}
 
  \noindent 
  \textbf{Key-words}: In vitro fertilization. Euploidy. Time-Lapse System. PGT-A. Artificial Intelligence. Machine Learning. Data analysis.
 \end{otherlanguage*}
\end{resumo}
