\begin{agradecimentos}

\section*{Maria Abritta e Sabrina Berno}

Gostaríamos de expressar nossa mais profunda gratidão a todos que, de alguma forma, contribuíram para a realização deste trabalho. Nossa trajetória foi marcada por desafios e aprendizados, mas também por momentos de superação, que só foram possíveis graças ao apoio e à inspiração de pessoas extraordinárias. Este projeto é o reflexo de um esforço coletivo e cada etapa foi enriquecida pela dedicação e pelo carinho de todos que nos acompanharam.

Primeiramente, nosso agradecimento ao professor Geroge Marsicano Correa, nosso orientador, que esteve ao nosso lado, nos impulsionando a superar nossos próprios limites e a acreditar em nosso potencial. Sua confiança em nossas capacidades, suas orientações sempre precisas e sua paciência ao longo de todo o processo foram fundamentais para que pudéssemos chegar onde estamos hoje. Não foi apenas um orientador técnico, mas também uma fonte constante de inspiração e motivação. Este trabalho reflete não apenas nosso esforço, mas também a influência de alguém que sempre acreditou em nós e nos guiou com sabedoria e generosidade. Somos imensamente gratas por termos tido o privilégio de trilhar esta jornada com ele ao nosso lado.

Também não podemos deixar de reconhecer a contribuição inestimável do Dr. Bruno Ramalho. Além de ser uma referência na medicina reprodutiva, o Dr. Bruno foi um pilar essencial para o desenvolvimento desta pesquisa. Agradecemos por disponibilizar os dados fundamentais para nossa análise, por compartilhar seu vasto conhecimento e por indicar leituras que enriqueceram profundamente nosso entendimento. Sua generosidade em estar sempre disponível, esclarecendo dúvidas e acompanhando o progresso do trabalho foi indispensável para o sucesso deste projeto.

Ao professor Fabiano Araujo Soares, nosso mais profundo agradecimento por sua orientação técnica. Suas ideias, apoio constante e contribuição na definição das abordagens de Inteligência Artificial foram cruciais para a estruturação e a fundamentação deste estudo. Sua expertise e dedicação nos proporcionaram segurança para tomar decisões estratégicas e aprimorar nosso trabalho.

Por fim, mas não menos importante, agradecemos aos nossos familiares e amigos. Vocês foram o porto seguro nos momentos mais desafiadores, oferecendo palavras de incentivo, compreensão e amor incondicional. O apoio emocional de vocês foi tão essencial quanto qualquer outro recurso técnico ou acadêmico e somos eternamente gratas por terem estado ao nosso lado.

A todos que, direta ou indiretamente, fizeram parte desta jornada, nosso mais sincero obrigado. Este trabalho não é apenas o resultado de nosso esforço, mas um reflexo da contribuição de cada um de vocês.

\section*{Maria Abritta}

Primeiramente, quero agradecer a Deus e Maria Santíssima, por me concederem tantas bênçãos e graças em minha vida. Eles sempre me guiam e fortalecem e sou imensamente grata por tudo que Eles fizeram e têm feito por mim.

Agradeço imensamente à minha família: meu pai, Norberto Abritta; minha mãe, Heleni Abritta; minha irmã, Victória Abritta; e ao meu namorado, Douglas de Oliveira. Vocês sempre estiveram ao meu lado, oferecendo apoio incondicional em todos os momentos. Suas palavras de incentivo e fé em mim são uma fonte constante de motivação, alimentando meu crescimento e me impulsionando a alcançar meu melhor. Obrigada por exaltarem minhas ideias, acreditarem nos meus sonhos e me encorajarem a nunca desistir. Sou eternamente grata por ter vocês como pilares na minha vida.

Não posso deixar de expressar minha profunda gratidão ao meu orientador, Geroge Marsicano, que esteve ao meu lado desde o início desta jornada. Sua orientação, paciência e dedicação foram essenciais para o desenvolvimento deste trabalho. Nunca esquecerei de suas palavras sábias que me ajudaram a encontrar clareza em momentos de dúvida.

Agradeço também aos meus amigos, que com tanto carinho ouviram, muitas vezes, sobre este TCC e todas as minhas descobertas. Sua paciência e apoio incondicional foram fundamentais para minha trajetória.

Por fim, agradeço à Universidade de Brasília, que sempre abriu portas para mim, me oferecendo um ambiente de aprendizado e desenvolvimento. Acredito que a educação realmente tem o poder de mudar vidas e sou grata por fazer parte dessa instituição.

\section*{Sabrina Berno}

Gostaria de expressar minha gratidão a todos que estiveram ao meu lado durante a realização deste trabalho. Cada um de vocês foi essencial para que eu pudesse alcançar este objetivo e, por isso, gostaria de agradecer por vocês estarem na minha vida.

Agradeço à minha família, que sempre me apoiou e incentivou a seguir meus sonhos. Destaco minha mãe, Hercília Berno, que sempre esteve ao meu lado, me apoiando em cada passo desta jornada. Seu amor, paciência e dedicação foram fundamentais para eu acreditar em mim mesma e seguir em frente. Mãe, sou profundamente grata por tudo o que fez por mim. Você sempre foi minha fonte de força e inspiração. Dedico este projeto ao meu pai, Carlos Berno, porque tenho certeza de que ele adoraria ler sobre isso e seria um dos meus maiores apoiadores em tudo que estou fazendo. Agradeço a todos os meus familiares por serem a base sólida sobre a qual construí minha trajetória, sempre com amor e dedicação.

Agradeço também aos meus amigos, tanto da área de medicina e de engenharia de software quanto de fora, que sempre foram meus melhores apoiadores. A amizade de vocês é um dos maiores presentes da minha vida. Agradeço às minhas amigas que me orientaram na busca por profissionais da área médica e me enviaram diversos artigos e informações que foram fundamentais para a realização deste trabalho.

Agradeço à minha dupla de TCC, Maria Abritta, que é a melhor pessoa para formar uma equipe que alguém poderia desejar. A maneira como você se entregou à tarefa e compartilhou sua energia tornou cada etapa mais leve e divertida. Este projeto é o reflexo da nossa parceria e da nossa amizade, que se fortaleceu ao longo deste processo.

Obrigada, Deus, por todas as oportunidades que surgiram na minha vida. Sou grata por cada passo dado, cada aprendizado adquirido e por tudo o que ainda está por vir.

\end{agradecimentos}
