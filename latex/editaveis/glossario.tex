\chapter*{GLOSSÁRIO}
\addcontentsline{toc}{chapter}{GLOSSÁRIO}

\begin{flushleft}

\begin{tabularx}{\textwidth}{p{4.5cm} X} % Usa toda a largura da página e duas colunas
    Aneuploidia & Condição em que o número de cromossomos não é múltiplo exato de 23. \\
    Blastocisto & Estágio do desenvolvimento embrionário, que ocorre cerca de 5 a 6 dias após a fecundação, apresentando uma cavidade interna e as primeiras divisões celulares mais complexas. \\
    Blastômeros & Células resultantes das divisões iniciais do embrião, multiplicando-se durante as primeiras fases de desenvolvimento. \\
    Biópsia & Procedimento médico para retirar uma pequena amostra de tecido ou células para análise. Utilizado em reprodução assistida para avaliar a saúde do embrião. \\
    Citoplasma & Parte do conteúdo celular que envolve o núcleo e onde ocorrem várias funções vitais da célula, como metabolismo e síntese de proteínas. \\
    Clivagem & Processo de divisão celular do embrião, onde uma célula inicial se divide sucessivamente em células menores chamadas blastômeros. \\
    Cromossomos & Estruturas presentes no núcleo celular que carregam a informação genética. Os humanos têm 23 pares de cromossomos, totalizando 46. \\
    Dados Morfocinéticos & Análise dos parâmetros morfológicos e cinéticos do embrião, como sua estrutura e o movimento/desenvolvimento ao longo do tempo. \\
    Euploidia & Condição em que o número de cromossomos é múltiplo exato de 23. \\
    Gravidez Clínica & Definição de uma gestação confirmada por ultrassonografia, com presença de embrião e batimento cardíaco fetal no útero. \\
    Implantação de Embrião & Processo no qual o embrião se fixa e se insere na parede do útero, iniciando o desenvolvimento da gravidez. \\
    Massa Celular Interna do Embrião & Conjunto de células do blastocisto que dará origem ao feto durante o desenvolvimento da gestação. \\
\end{tabularx}

\begin{tabularx}{\textwidth}{p{4.5cm} X} % Usa toda a largura da página e duas colunas
    Mosaico & Tipo de aneuploidia em que algumas células têm o número correto de cromossomos, enquanto outras têm um número alterado. \\
    Ploidias & Refere-se ao número de conjuntos de cromossomos em uma célula ou organismo. \\
    Trofectoderma & Camada externa do blastocisto que dará origem à placenta, sendo essencial para a implantação do embrião no útero. \\
\end{tabularx}

\end{flushleft}
