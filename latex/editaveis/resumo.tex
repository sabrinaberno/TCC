\begin{resumo}
A fertilização in vitro é amplamente utilizada por casais com dificuldades reprodutivas ou que desejam postergar a gravidez, mas seu sucesso depende de vários fatores, sendo a seleção de embriões viáveis um dos mais críticos. Escolher o embrião correto aumenta as chances de implantação, reduz o risco de abortos e melhora as taxas de nascimentos saudáveis. O Teste Genético Pré-Implantacional para Aneuploidia é o método mais usado, mas, além do custo elevado, apresenta limitações de precisão diagnóstica, podendo levar à transferência de embriões inviáveis ou exclusão de viáveis. Com o avanço de tecnologias como o Time-Lapse System, que captura imagens contínuas do desenvolvimento embrionário, surge a chance de prever a euploidia usando dados morfocinéticos de forma menos invasiva e mais eficaz. A pesquisa adota uma abordagem quantitativa e experimental, focando na análise de dados numéricos de padrões morfocinéticos de embriões para o desenvolvimento da IA preditiva. Este trabalho desenvolve uma abordagem baseada em Machine Learning para identificar padrões nos dados morfocinéticos e prever a euploidia, oferecendo uma alternativa ao PGT-A sem intervenções invasivas. Após revisão de literatura e análise de correlação, foram identificadas as variáveis mais relevantes, sendo a idade e tb-t2b as com maior influência negativa. Os modelos de Machine Learning foram treinados e validados utilizando métricas como acurácia, sensibilidade e especificidade, de forma a avaliar seu desempenho na predição de euploidia a partir de dados morfocinéticos. O modelo desenvolvido, baseado em uma Rede Neural Artificial do tipo Perceptron Multicamadas (MLP), alcançou uma acurácia de 88,2\% e uma Área sob a Curva ROC (AUC) de 0,944, indicando elevada capacidade discriminativa entre embriões euploides e aneuploides. A avaliação por meio da matriz de confusão evidenciou uma excelente performance na identificação de embriões aneuploides, com recall perfeito (1,00) para essa classe e robusto (0,75) para embriões euploides, sem ocorrência de falsos positivos. A integração do método LIME possibilitou tornar as predições do modelo interpretáveis, identificando quais variáveis morfocinéticas mais influenciaram cada predição individual, uma característica fundamental para garantir transparência e confiabilidade no contexto biomédico. A solução desenvolvida oferece uma ferramenta promissora para complementar o PGT-A, reduzindo custos e evitando intervenções invasivas, com potencial de aplicação real na rotina clínica da reprodução assistida.

\vspace{\onelineskip}
    
\noindent
\textbf{Palavras-chave}: Fertilização in vitro. Euploidia. Time-Lapse System. PGT-A. Inteligência Artificial. Aprendizado de máquina. Análise de dados.

\end{resumo}
