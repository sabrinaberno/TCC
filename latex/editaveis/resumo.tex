\begin{resumo}
A fertilização in
 vitro é amplamente utilizada para casais com dificuldades reprodutivas ou com o desejo de postergar a gravidez, mas ainda enfrenta desafios significativos, principalmente em relação à seleção de embriões viáveis. Aproximadamente 50\% das perdas gestacionais estão associadas a anomalias cromossômicas, tornando a escolha de embriões euploides essencial para aumentar as taxas de sucesso do procedimento. O Teste Genético Pré-Implantacional para Aneuploidia é atualmente o método mais utilizado para essa seleção, mas envolve a biópsia das células do embrião, um procedimento invasivo que pode prejudicar a viabilidade embrionária, além de possuir um custo elevado. Com o avanço das tecnologias de monitoramento, como o Time-Lapse System, que captura imagens contínuas do desenvolvimento embrionário, surge a possibilidade de usar dados morfocinéticos para prever a euploidia de forma menos invasiva e mais eficaz. Este trabalho tem como objetivo desenvolver uma abordagem baseada em inteligência artificial para identificar padrões nos dados morfocinéticos de embriões obtidos pelo Time-Lapse System, com o intuito de prever a porcentagem de euploidia e fornecer uma alternativa ao PGT-A sem a necessidade de intervenções invasivas. A metodologia adotada para o estudo segue uma abordagem quantitativa, baseada na revisão de literatura e na análise de correlações entre variáveis morfocinéticas dos embriões. A partir de uma revisão de literatura, foi possível identificar as variáveis mais relevantes para a análise, e, em seguida, realizamos a análise de correlação entre essas variáveis. Os resultados encontrados nas correlações confirmaram as tendências observadas na literatura, mostrando que as variáveis morfocinéticas selecionadas possuem uma relação significativa com a previsão de euploidia. Embora esta fase inicial se concentre na análise de correlação, ela fornece uma base sólida para o desenvolvimento de modelos de aprendizado de máquina mais complexos que poderão ser aplicados para prever a euploidia de forma mais precisa. O processo inclui o pré-processamento dos dados, extração de características morfocinéticas relevantes, treinamento dos modelos de IA, e validação dos resultados com métricas como acurácia, sensibilidade e especificidade. A comparação entre os resultados obtidos pela IA e o PGT-A servirá para avaliar a eficácia e a precisão da abordagem proposta. Espera-se que a inteligência artificial, ao analisar os dados morfocinéticos, apresente uma alta acurácia na predição de euploidia, superior a 70\%, com potencial para complementar os métodos invasivos atualmente utilizados e auxiliar os médicos na escolha do melhor embrião. A aplicação dessa solução pode resultar em um aumento significativo nas taxas de sucesso da FIV, ao mesmo tempo que proporciona uma abordagem mais econômica e menos prejudicial para os embriões. Este estudo tem o potencial de transformar as práticas de reprodução assistida, reduzindo custos e riscos associados a métodos invasivos, e oferecendo uma solução mais acessível para a escolha dos embriões com maior probabilidade de sucesso na implantação.

 \vspace{\onelineskip}
    
\noindent
\textbf{Palavras-chave}: Fertilização in vitro. Euploidia. Time-Lapse System. PGT-A. Inteligência Artificial. Aprendizado de máquina. Análise de dados.

\end{resumo}
