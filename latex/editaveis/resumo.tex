\begin{resumo}
A fertilização in vitro é amplamente utilizada por casais com dificuldades reprodutivas ou que desejam postergar a gravidez, mas seu sucesso depende de vários fatores, sendo a seleção de embriões viáveis um dos mais críticos. Escolher o embrião correto aumenta as chances de implantação, reduz o risco de abortos e melhora as taxas de nascimentos saudáveis. O Teste Genético Pré-Implantacional para Aneuploidia é o método mais usado, mas, além do custo elevado, apresenta limitações de precisão diagnóstica, podendo levar à transferência de embriões inviáveis ou exclusão de viáveis. Com o avanço de tecnologias como o Time-Lapse System, que captura imagens contínuas do desenvolvimento embrionário, surge a chance de prever a euploidia usando dados morfocinéticos de forma menos invasiva e mais eficaz. A pesquisa adota uma abordagem quantitativa e experimental, focando na análise de dados numéricos de padrões morfocinéticos de embriões para o desenvolvimento da IA preditiva. Este trabalho desenvolve uma abordagem baseada em Machine Learning para identificar padrões nos dados morfocinéticos e prever a euploidia, oferecendo uma alternativa ao PGT-A sem intervenções invasivas. Após revisão de literatura e análise de correlação, foram identificadas as variáveis mais relevantes, sendo a idade e tb-t2b as com maior influência negativa. Os próximos passos incluem treinar os modelos de Machine Learning e validar os resultados com métricas como acurácia, sensibilidade e especificidade. A comparação com o PGT-A avaliará a eficácia da abordagem. Espera-se que a análise dos dados morfocinéticos com Machine Learning alcance alta acurácia, superior a 70\%, complementando os métodos invasivos e auxiliando na escolha do melhor embrião. A aplicação dessa solução pode contribuir com pesquisas para aumentar as taxas de sucesso da FIV, oferecendo uma alternativa mais econômica e menos prejudicial.
\vspace{\onelineskip}
    
\noindent
\textbf{Palavras-chave}: Fertilização in vitro. Euploidia. Time-Lapse System. PGT-A. Inteligência Artificial. Aprendizado de máquina. Análise de dados.

\end{resumo}
